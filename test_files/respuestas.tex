\documentclass{article}
\usepackage[spanish]{babel}
\usepackage[utf8]{inputenc}
\usepackage[T1]{fontenc}
\usepackage[letterpaper,top=2cm,bottom=2cm,left=3cm,right=3cm,marginparwidth=1.75cm]{geometry}
\begin{document}
\section{Parte 1}
\subsection{Tema}
La reunión se centró en la aprobación del orden del día, la revisión del acta anterior y la propuesta de un nuevo grupo de trabajo en relación con la inteligencia artificial. Se discutió la decisión tomada en una reunión anterior, liderada por el rector, sobre el enfoque de la universidad en temas de inteligencia artificial.

\subsection{Agenda}
Aprobación del orden del día, revisión del acta anterior, discusión sobre la propuesta de un grupo de trabajo en inteligencia artificial, decisión de la rectoría sobre los temas de inteligencia artificial.

\subsection{Decisiones}
Aprobación del orden del día, decisión de la rectoría de que los temas de inteligencia artificial serán "acunados" en la facultad de ciencias humanidades, posibilidad de presentar la propuesta del grupo de trabajo en inteligencia artificial en el próximo consejo.

\subsection{Compromisos}
Revisar el acta anterior y proporcionar comentarios, determinar si la propuesta del grupo de trabajo en inteligencia artificial es conocida por el doctor que la propone, si no es así, programar una reunión para revisar el tema antes de presentarlo al consejo. Si la propuesta es conocida, presentarla en el próximo consejo.
\section{Parte 2}
\subsection{Tema}
En esta parte de la reunión, se discutió la presentación de una propuesta por parte de los ingenieros Esteban, de la línea y Carlos Mauricio. También se habló sobre la presentación de propuestas de un grupo de trabajo de leyes artificiales y la socialización de nuevas selectivas de química y ambiental. Además, se mencionó la intervención de los representantes de estudiantes, docentes e ingresados, y la intervención del representante del cuerpo de regímenes. Finalmente, se discutió el proceso de preparación de las visitas.

\subsection{Agenda}
La agenda de la reunión incluyó la presentación de la propuesta de los ingenieros, la presentación de propuestas de grupo de trabajo de leyes artificiales, la socialización de nuevas selectivas de química y ambiental, la intervención de los representantes de estudiantes, docentes e ingresados, la intervención del representante del cuerpo de regímenes y el proceso de preparación de las visitas.

\subsection{Decisiones}
Las decisiones tomadas durante la reunión incluyeron la aprobación de la presentación de la propuesta de los ingenieros, la aprobación de la presentación de las propuestas de grupo de trabajo de leyes art
\section{Parte 3}
\subsection{Tema}
La reunión se centró en la organización de un evento para estudiantes de Química y carreras afines con el objetivo de aumentar la interacción, colaboración y aprendizaje en temas de investigación. El evento también busca estimular la creatividad y la innovación, promover la conciencia ambiental y la sustentabilidad en la ingeniería química, y brindar oportunidades de desarrollo profesional y vinculación con la industria.
\subsection{Agenda}
Los temas principales de la agenda incluyen: sustentabilidad energética y ambiental, aplicaciones ingenieras a partir de bio, micro y nanotecnología, innovación de procesos a partir de fuentes convencionales y no convencionales, ética, control y prevención en la industria y aprendizaje ingeniero. El evento se llevará a cabo del lunes 26 de junio al lunes 30 de junio, con actividades que incluyen ponencias, talleres, paneles y conversatorios.
\subsection{Decisiones}
Las decisiones más importantes incluyen: el enfoque del evento en carreras relacionadas con la química, incluyendo petróleo y ambiental, la organización de diferentes comités (publicidad, académico, visitas industriales, logístico, financiero,
\section{Parte 4}
\subsection{Tema}
El tema principal de la reunión fue la organización y planificación del 33º encuentro vinculado desde 1990, que incluirá invitados de Estados Unidos, México, Perú y Noruega. Se discutieron los detalles de las visitas industriales, las actividades culturales y los recorridos turísticos.

\subsection{Agenda}
Los temas más importantes mencionados fueron: la confirmación de las visitas industriales, la organización de las actividades culturales, la planificación de los recorridos turísticos, la coordinación con la agenda del rector y la preparación del auditorio para la inauguración.

\subsection{Decisiones}
Las decisiones más importantes tomadas fueron: confirmar las ocho visitas industriales (Laboratorio Electro, Herpo, Agricultores de Cundinamarca, Pelican, PTAB, Coca-Cola, Vinigo, Rano), realizar una cena de inauguración y un recorrido por la calera, y finalizar con una cena de gala.

\subsection{Compromisos}
Los compromisos establecidos fueron: coordinar con el rector para la apertura del encuentro, asegurar el auditorio para la inauguración, y organizar los recorridos turísticos para el domingo 25 y el domingo 2 de Julio.
\section{Parte 5}
\subsection{Tema}
La reunión se centró en la discusión de la agenda de acreditación, la organización de eventos y cierres, la inclusión del museo en la agenda, y la importancia del comportamiento y el autocuidado durante las actividades extracurriculares. Se abordó también el tema de la certificación de los participantes en el evento.

\subsection{Agenda}
Agenda de acreditación, organización de eventos y cierres, inclusión del museo en la agenda, comportamiento y autocuidado durante las actividades extracurriculares, proceso de certificación de los participantes.

\subsection{Decisiones}
Revisar la agenda de acreditación en marzo, incluir el museo en la agenda de visitas, emitir alertas sobre el comportamiento y autocuidado durante las actividades extracurriculares, entregar certificados de participación a los participantes del evento.

\subsection{Compromisos}
Revisar la agenda de acreditación en marzo (Responsable: Comité Operativo de Acreditación), Incluir el museo en la agenda de visitas (Responsable: Organizadores del evento), Emitir alertas sobre el comportamiento y autocuidado durante las actividades extracurriculares (Responsable: Todos los participantes), Entregar certificados de participación
\section{Parte 6}
\subsection{Tema}
Reunión en la que se discutieron temas relacionados con la firma del director de investigaciones, la visibilidad del evento, la ruta de investigación científica, la acreditación del programa de ingeniería de petróleo, la imagen de la universidad, el prestigio de los eventos, el compromiso de los participantes, y una propuesta del grupo de trabajo en temas de inteligencia artificial.

\subsection{Agenda}
Firma del director de investigaciones, visibilidad del evento, ruta de investigación científica, acreditación del programa de ingeniería de petróleo, imagen de la universidad, prestigio de los eventos, compromiso de los participantes, propuesta del grupo de trabajo en temas de inteligencia artificial.

\subsection{Decisiones}
La decisión de que los temas de data, ciencia y seguridad en inteligencia artificial son del resolver la facultad de ciencias y humanidades, la decisión de que la presentación la escuche dejando Carlos en la audición para que todo nos quede alineados institucionalmente.

\subsection{Compromisos}
Compromiso de los participantes para mantener la imagen y el prestigio de la universidad, compromiso del grupo de trabajo en temas de inteligencia artificial para presentar su propuesta. No se establecieron responsables y fe
\section{Parte 7}
\subsection{Tema}
La reunión se centró en la presentación de una propuesta relativa a la inteligencia artificial y las vacunas en la facultad de ciencias y humanidades. Se discutió la integración de diferentes facultades y programas en este proyecto, con el objetivo de alinear los esfuerzos y asegurar que todas las partes estén en la misma página. Se mencionó también la inversión en la sala COCO y la posibilidad de ofrecer cursos de capacitación.

\subsection{Agenda}
Presentación de propuesta sobre inteligencia artificial y vacunas, Discusión sobre la integración de diferentes facultades y programas, Inversión en la sala COCO, Posibilidad de ofrecer cursos de capacitación.

\subsection{Decisiones}
Se decidió seguir adelante con la propuesta presentada, Integrar a la facultad de arquitectura en el proyecto, Realizar cursos de capacitación, Utilizar la sala COCO para el desarrollo del proyecto.

\subsection{Compromisos}
El comité editorial se comprometió a alinear sus esfuerzos con los de la propuesta, El profe John Ramírez se comprometió a integrar la facultad de arquitectura en el proyecto, Se acordó organizar cursos de capacitación, Se acordó utilizar la sala COCO para el
\section{Parte 8}
\subsection{Tema}
La reunión se centró en la discusión sobre la integración de la inteligencia artificial en varios proyectos y cómo esto puede beneficiar a la comunidad universitaria y al sector externo. Se discutió la posibilidad de ofrecer servicios a empresas y al sector universitario a través de la sala COCO. Se habló de un proyecto de gemelos digitales desarrollado en CEPIS y se mencionó la importancia de la formación docente en el área de la programación.

\subsection{Agenda}
Integración de la inteligencia artificial, Ofrecer servicios al sector externo, Proyecto de gemelos digitales en CEPIS, Capacitación docente en programación, Importancia de la sala COCO.

\subsection{Decisiones}
Inversiones en capacidades docentes, Desarrollo de un proyecto de gemelos digitales en CEPIS, Apertura de la sala COCO para servicios a la comunidad y empresas, Integración de profesores en la planta de Huawei.

\subsection{Compromisos}
El montaje y la finalización de la sala COCO, Responsable: Equipo de CEPIS, Fecha: No especificada, Capacitación docente en programación, Responsable: Directora de Huawei, Fecha: No especificada, Desarrollo del proyecto de gem
\section{Parte 9}
\subsection{Tema}
La reunión se centró en la expansión de la consultoría a través de medios digitales, la integración con el CEPIS para llegar al sector externo, el fortalecimiento del Centro de Mundo Actual, la visión populacional y la implementación de la inteligencia artificial en la educación.

\subsection{Agenda}
Medios digitales y consultoría, Integración con el CEPIS, Fortalecimiento del Centro de Mundo Actual, Implementación de la inteligencia artificial en la educación, Formación de un nuevo grupo de investigación.

\subsection{Decisiones}
Se decidió fortalecer el Centro de Mundo Actual, implementar la inteligencia artificial en la educación, y formar un nuevo grupo de investigación interdisciplinario que abordará varias líneas de investigación a nivel universitario.

\subsection{Compromisos}
Se acordó que los docentes de diferentes facultades formarán un nuevo grupo de investigación interdisciplinario. Se espera que este grupo de investigación tenga varias líneas de investigación a nivel universitario. También se acordó fortalecer el Centro de Mundo Actual e implementar la inteligencia artificial en la educación. No se establecieron fechas específicas para estos compromisos.
\section{Parte 10}
\subsection{Tema}
La reunión se centró en la discusión sobre la contribución de los semilleros de investigación a los grupos de investigación existentes, la creación de un nuevo ambiente de colaboración, y la utilización de la infraestructura existente para fomentar la investigación.

\subsection{Agenda}
Contribución de los semilleros a los grupos de investigación, creación de un nuevo ambiente de colaboración, aprovechamiento de la infraestructura de investigación existente, articulación de la estructura de investigación, desarrollo de proyectos desde la formación investigativa, exploración de nuevas líneas de investigación.

\subsection{Decisiones}
Se decidió que los semilleros contribuirán a los grupos de investigación, se creará un nuevo ambiente de colaboración que sea transversal a los grupos, se aprovechará la infraestructura de investigación existente, se articulará la estructura de investigación desde la formación hasta la aplicación, y se explorarán nuevas líneas de investigación.

\subsection{Compromisos}
Se estableció el compromiso de los semilleros para contribuir a las líneas de investigación de los grupos, se comprometió a crear un nuevo ambiente de colaboración, se comprometió a aprovechar la infraestructura de investigación existente y a articular la estructura de investigación. No se establecieron
\section{Parte 11}
\subsection{Tema}
Durante esta parte de la reunión se discutió la creación de nuevos programas y maestrías, la identificación de líderes para cada temática y la estrategia para implementar los programas de incubadora. Se mencionó la importancia de la formación en investigación y la integración de diferentes facultades.

\subsection{Agenda}
Creación de 14 maestrías nuevas, Implementación de programas de incubadora, Identificación de líderes para cada temática, Formación en investigación, Integración de diferentes facultades.

\subsection{Decisiones}
Se decidió crear 14 maestrías nuevas y programas de incubadora, Identificar líderes para cada temática, enfocarse en la formación en investigación y la integración de diferentes facultades.

\subsection{Compromisos}
Se estableció el compromiso de crear 14 maestrías nuevas y programas de incubadora, Identificar líderes para cada temática, enfocarse en la formación en investigación y la integración de diferentes facultades. No se establecieron fechas específicas ni responsables para cada compromiso.
\section{Parte 12}
\subsection{Tema}
La reunión se centró en la discusión de las estrategias institucionales, la planificación de trabajo y la asignación de roles en relación con los programas nuevos por nacer. Se discutieron temas como la postulación, la edición de asignaturas, la incorporación de talento y la colaboración con aliados.

\subsection{Agenda}
Identificación por el tema postulaciones, Planes de trabajo, Asignación de roles a las directoras María Angélica, Nubia y Mónica, Proyecto del curador, Necesidad de talento para las asignaturas, Proceso de incubación, Condiciones de investigación y extensión, Propuestas para la educación continua, Proyecto con la entidad aliada, Tema de la educación y la ilustración, Representaciones seleccionadas.

\subsection{Decisiones}
Se decidió que las directoras María Angélica, Nubia y Mónica estarán a cargo de las asignaturas, Se acordó que se necesita traer más talento para las asignaturas, Se decidió que el proceso de incubación ya está establecido, Se decidió que se necesita trabajar más en la condición de investigación y extensión, Se decidió que las propuestas para la educación continua se alinearán con el proyecto con
\section{Parte 13}
\subsection{Tema}
La reunión abordó la necesidad de finalizar el acta de la facultad de ingeniería, la importancia de tener un acta como evidencia, y la planificación de los trabajos futuros para generar valor. También se discutió la seguridad en la ciudad y la posibilidad de que los jóvenes obtengan una vida en la ciudad.

\subsection{Agenda}
Finalización del acta de la facultad de ingeniería, Importancia de las actas como evidencia, Planificación de trabajos futuros para generar valor, Seguridad en la ciudad, Posibilidad de vida en la ciudad para los jóvenes.

\subsection{Decisiones}
La necesidad de finalizar el acta de la facultad de ingeniería, Importancia de tener actas como evidencia, Planificación de los trabajos futuros para generar valor.

\subsection{Compromisos}
No se establecieron compromisos en esta parte de la reunión.

\end{document}